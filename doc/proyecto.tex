%%%%%%%%%%%%%%%%%%%%%%%%%%%%%%%%%%%%%%%%%
% Short Sectioned Assignment LaTeX Template Version 1.0 (5/5/12)
% This template has been downloaded from: http://www.LaTeXTemplates.com
% Original author:  Frits Wenneker (http://www.howtotex.com)
% License: CC BY-NC-SA 3.0 (http://creativecommons.org/licenses/by-nc-sa/3.0/)
%%%%%%%%%%%%%%%%%%%%%%%%%%%%%%%%%%%%%%%%%

% \documentclass[paper=a4, fontsize=11pt]{scrartcl} % A4 paper and 11pt font size
\documentclass[11pt, a4paper]{book}
\usepackage[utf8]{inputenc}
\usepackage[T1]{fontenc} % Use 8-bit encoding that has 256 glyphs
\usepackage[spanish, es-tabla]{babel} % Selecciona el español para palabras introducidas automáticamente, p.ej. "septiembre" en la fecha y especifica que se use la palabra Tabla en vez de Cuadro
\usepackage{fourier} % Use the Adobe Utopia font for the document - comment this line to return to the LaTeX default
\usepackage{listings} % para insertar código con formato similar al editor
\usepackage{url} % ,href} %para incluir URLs e hipervínculos dentro del texto (aunque hay que instalar href)
\usepackage{graphics,graphicx, float} %para incluir imágenes y colocarlas
\usepackage[gen]{eurosym} %para incluir el símbolo del euro
\usepackage{cite} %para incluir citas del archivo <nombre>.bib
\usepackage{enumerate}
\usepackage{hyperref}
\usepackage{graphicx}
\usepackage{tabularx}
\usepackage{booktabs}
\usepackage{subfigure}

\usepackage[table,xcdraw]{xcolor}
\hypersetup{
	colorlinks=true,	% false: boxed links; true: colored links
	linkcolor=black,	% color of internal links
	urlcolor=cyan		% color of external links
}
\renewcommand{\familydefault}{\sfdefault}
\usepackage{fancyhdr} % Custom headers and footers
\pagestyle{fancyplain} % Makes all pages in the document conform to the custom headers and footers
\fancyhead[L]{} % Empty left header
\fancyhead[C]{} % Empty center header
\fancyhead[R]{Rubén Delgado Pareja} % My name
\fancyfoot[L]{} % Empty left footer
\fancyfoot[C]{} % Empty center footer
\fancyfoot[R]{\thepage} % Page numbering for right footer
%\renewcommand{\headrulewidth}{0pt} % Remove header underlines
\renewcommand{\footrulewidth}{0pt} % Remove footer underlines
\setlength{\headheight}{13.6pt} % Customize the height of the header

\usepackage{titlesec, blindtext, color}
\definecolor{gray75}{gray}{0.75}
\newcommand{\hsp}{\hspace{20pt}}
\titleformat{\chapter}[hang]{\Huge\bfseries}{\thechapter\hsp\textcolor{gray75}{|}\hsp}{0pt}{\Huge\bfseries}
\setcounter{secnumdepth}{4}
\usepackage[Lenny]{fncychap}


\begin{document}

% Plantilla portada UGR
\input{portada/portada}

% Plantilla prefacio UGR
\thispagestyle{empty}

\begin{center}
	{\large\bfseries Triage Inteligente de Consulta Médica \\ Sistema de citas priorizado por el riesgo del paciente }\\
\end{center}
\begin{center}
	Rubén Delgado Pareja\\
\end{center}

%\vspace{0.7cm}

\vspace{0.5cm}
\noindent\textbf{Palabras clave}: \textit{Software libre, Triage, Metodología Ágil, Diseño Dirigido por el Dominio,}
\vspace{0.7cm}

\noindent\textbf{Resumen}\\
Debido al precario estado de la sanidad española las colas de citas médicas y urgencias están saturadas. Aprovechando el potencial de los sistemas informáticos se pretende minimizar las
consecuencias negativas de estas largas esperas usando el método del triage para priorizar a los pacientes más graves. Para ello se propone un sistema de citas médicas que asigna turnos en función de la gravedad del paciente
Siguiendo la metodología ágil se ha implementado una API Rest que permita registrar pacientes, almacenar su información médica a través de un formulario basado en el protocolo de triage y asignarles un turno en función de su nivel de prioridad.

\cleardoublepage

\begin{center}
	{\large\bfseries Intelligent Triage for Medical Consultation \\ Appointment system prioritized by patient risk}\\
\end{center}
\begin{center}
	Rubén Delgado Pareja\\
\end{center}
\vspace{0.5cm}
\noindent\textbf{Keywords}: \textit{Open source, Triaje, Agile methodology, Domain Driven Design}, \textit{floss}
\vspace{0.7cm}

\noindent\textbf{Abstract}\\
Due to the precarious state of the Spanish healthcare system, queues for medical appointments and emergencies are saturated. By exploiting the potential of IT systems, the aim is to minimise the negative consequences of long waiting times by using the
negative consequences of these long waits by using the triage method to prioritise the most serious patients. To this end, a system of medical appointments is proposed that assigns shifts according to the severity of the patient.
Following the agile methodology, an API Rest has been implemented to register patients, store their medical information through a form based on the triage protocol and assign them a shift according to their priority level.

\cleardoublepage

\thispagestyle{empty}

\noindent\rule[-1ex]{\textwidth}{2pt}\\[4.5ex]

D. \textbf{Juan Julián Merelo Guervós}, Profesor del departamento de Arquitectura y Tecnología de Computadores.

\vspace{0.5cm}

\textbf{Informo:}

\vspace{0.5cm}

Que el presente trabajo, titulado \textit{\textbf{Triage Inteligente de Consulta Médica. Sistema de citas priorizado por el riesgo del paciente}},
ha sido realizado bajo mi supervisión por \textbf{Rubén Delgado Pareja}, y autorizo la defensa de dicho trabajo ante el tribunal
que corresponda.

\vspace{0.5cm}

Y para que conste, expiden y firman el presente informe en Granada a Septiembre de 2024.

\vspace{1cm}

\textbf{El/la director(a)/es: }

\vspace{5cm}

\noindent \textbf{Juan Julián Merelo Guervós}

\chapter*{Agradecimientos}

Primero de todo, quiero agradecérmelo a mi mismo, por ser capaz de llevar a cabo este proyecto. A pesar del poco tiempo que he podido invertirle por temas laborales.
Después quería agradecer a toda la comunidad de informáticos, ya que son la mejor comunidad que he conocido jamás, de la mano también debo de agradecer al software libre.
Sin el software libre estaríamos limitando el conocimiento y la tecnología solamente a personas con recursos económicos.
Finalmente, agradecer a mi familia, amigos y compañeros de trabajo por su apoyo y comprensión durante la realización de este proyecto.


% Índice de contenidos
\newpage
\tableofcontents

% Índice de imágenes y tablas
\newpage
\listoffigures

% Si hay suficientes se incluirá dicho índice
\listoftables
\newpage

% Introducción
\chapter{Introducción}

Este proyecto es software libre, y está liberado con la licencia \cite{gplv3}.

\section{Motivación}
Debido a la crisis económica del 2008 la sanidad española no ha parado de recibir recortes, puede observarse en este
\href{https://www.consalud.es/politica/decada-recortes-2009-2018-efectos-infrafinanciacion-sanidad_87083_102.html}{artículo} donde
se estudia la financiación de la sanidad desde entonces. Esto provoca que se disminuya el número de sanitarios en España.
Esta situación que se lleva dando desde hace más de una década produce una sobrecarga de trabajo para los sanitarios y un retardo
en todas las listas de esperas, tanto en cirugías como en la atención primaria.
Con la llegada de la pandemia y del covid-19 todo empeoró llegando a colapsar totalmente la atención primaria debido a la demanda
de citas previas y a la escasez de profesionales enfocados a este servicio, ya que, también desde hace años los sanitarios se aglomeran
en los Hospitales por mejores situaciones laborales y salarios. En las siguientes links podemos observar la situación que he descrito.
\begin{itemize}
    \item \url{https://elpais.com/sociedad/2021-02-08/la-pandemia-sume-a-la-atencion-primaria-en-una-saturacion-permanente.html}
    \item \url{https://www.lavanguardia.com/vida/20211225/7952109/atencion-primaria-colapso-pacientes-medico-dias-espera.html}
    \item \url{https://www.lavozdegalicia.es/noticia/galicia/2022/01/16/dejando-atender-casos/0003_202201G16P2993.htm}
\end{itemize}
El problema que trato de abordar se escapa a mis posibilidades, sin embargo, mediante la herramienta que voy a desarrollar
durante el proyecto podré ofrecer una gran ayuda para los sanitarios centrados en la atención primaria.
Mi principal motivación es aportar mediante el uso de sistemas informáticos mi grano de arena en este gran problema con el que
tenemos que lidiar a día de hoy, construyendo una aplicación de citas médicas el cual tiene integrada un sistema de triage.

\section{Objetivos}

\textbf{Abstraer el concepto de triage. Automatizarlo y aplicarlo a otros ámbitos}. El sistema informático se basará en un triage de consultas médicas; el triage en el ámbito
de la sanidad es un proceso que permite gestionar el flujo de los pacientes teniendo en cuenta el riesgo cuando la demanda y necesidades clínicas
superan los recursos, es muy importante en urgencias clínicas \href{https://scielo.isciii.es/scielo.php?script=sci_arttext&pid=S1137-66272010000200008}{Aquí}
se puede encontrar más información.

El objetivo principal consiste en aprovechar el potencial del triage ya existente en urgencias y explotarlo al máximo, de tal forma que
consigamos aportar una nueva herramienta a los sanitarios con la intención de paliar la saturación de la sanidad en general y sus efectos negativos sobre los pacientes.
El triage inteligente consistirá en una abstracción del triage que ya se aplica en urgencias y automatizarlo en un sistema informático, incluso aplicarlo
otros ámbitos de la sanidad para mejorar su eficacia y eficiencia.

Los objetivos genéricos de un producto software son que pueda desplegarse en la nube, que pueda ser accesible por cualquier dispositivo que acceda a la aplicación web,
que mantenga la seguridad y privacidad de los datos de los pacientes y sanitarios y que la experiencia de usuario sea lo más satisfactoria posible.



% Descripción del problema y hasta donde se llega

\chapter{Descripción del problema}
Desde el inicio de la pandemia, atender telemáticamente es fundamental para no saturar las consultas médicas, sin embargo si todo
se realiza de forma manual la saturación puede ser la misma. Mi proyecto intentará resolver este problema mediante un sistema de
triage el cual es capaz de filtrar solicitudes de atención médica con distintos grados de urgencias y sea capaz de proporcionar
una ayuda a los profesionales en caso necesario


% Estado del arte
% 	1. Crítica al estado del arte
% 	2. Propuesta


\chapter{Estado del arte}

El software libre y sus licencias \cite{gplv3} ha permitido llevar a cabo una expansión del
aprendizaje de la informática sin precedentes.



\chapter{Metodología}
\label{ch:metodología}
En este capítulo se describirá la forma en la que se ha trabajado durante el desarrollo del proyecto.
El marco del desarrollo será el desarrollo ágil \cite{agile-software-development}; que se centra en el usuario, por lo que se utilizará una planificación ágil, con el objetivo
de conseguir un software de calidad y flexible. Esto debe estar presente constantemente en el desarrollo del proyecto.

\section{Planificación}
\label{sc:planificación}
La planificación que se va a emplear está basada en los {\href{https://agilemanifesto.org/iso/es/principles.html}{principios del manifiesto ágil}}
Buscamos un tipo de planificación que se centre en el usuario, para que el propio usuario nos aporte retroalimentación, aunque para este caso no sea posible por la
ausencia de un cliente real, por lo que debemos de escoger una planificación específica. En mi caso he usado Kanban con la ayuda del tablero de GitHub.

\begin{figure}[h]
    \centering
    \includegraphics[width=0.9\linewidth]{logos/kanban.png}
    \caption{Tablero Kanban.}
    \label{fig:layout1}
\end{figure}

Una planificación ágil nos permite adaptarnos al cambio de requisitos haciendo uso de las iteraciones y aprovechar las nuevas oportunidades que aparezcan.
El desarrollo ágil construye el sistema por medio de aportaciones frecuentes de código que aportan valor
al usuario, es necesario organizar las funcionalidades en bloques y que los cambios siempre surgen de una necesidad del usuario.

\section{Análisis de usuarios}
Dentro del marco de desarrollo ágil, comprender las necesidades y experiencias de los usuarios es fundamental para el éxito del proyecto.
En esta sección, nos adentramos en el concepto de User Journeys, una herramienta para entender cómo interactúan los usuarios con nuestro producto o servicio a lo largo de su ciclo de vida.


\section{\textit{User Journeys}}\label{anexo}

\subsection{Rafael González Pérez}
\begin{itemize}
    \item \textbf{Contexto de uso: }
          \begin{itemize}
              \item \textbf{¿Cuándo utiliza el ordenador?: }  Utiliza el ordenador con frecuencia
              \item \textbf{¿Dónde utiliza el ordenador?: } En el trabajo y en casa
              \item \textbf{¿Qué tipo de ordenador utiliza?: } El ordenador de la consulta de urgencias
          \end{itemize}
    \item \textbf{Misión: }
          \begin{itemize}
              \item \textbf{¿Para qué quiere utilizar nuestra aplicación?: } Como sanitario de urgencia quiero priorizar los pacientes con mayor riesgo
              \item \textbf{¿Qué espera encontrar en ella?: } Una aplicación web con el triage de urgencias que organice los turnos de los pacientes
          \end{itemize}
    \item \textbf{Motivación: }
          \begin{itemize}
              \item \textbf{¿Para cuándo quiere utilizarla?: } Cuando comience su jornada laboral en el hospital
              \item \textbf{¿Por qué quiere alcanzar ese objetivo?: } Para automatizar el proceso del triage
          \end{itemize}
    \item \textbf{Actitud hacia la tecnología: } Se siente cómodo navegando por Internet y usando las nuevas tecnologías. No le es ningún impedimento
\end{itemize}

\subsection{Escenario 1 de Rafael González Pérez}
Rafael es un sanitario que trabaja de guardias en un hospital de urgencias. Rafael se encarga
de gestionar los pacientes que llegan a urgencias y a través de una pequeña entrevista con el paciente asignarle
una prioridad u orden de consulta. Rafael descubre que en otros hospitales utilizan un sistema de triage inteligente
que les ayuda a priorizar a los pacientes y a organizar los turnos de los pacientes. Rafael propone implantarlo en su hospital.
Rafa consigue implantar el nuevo triage inteligente y lo pone en práctica en su jornada laboral. Los pacientes que van llegando
a urgencias realizan el cuestionario del triage y automáticamente se les asigna un nivel de prioridad y un tiempo de demora estimado.
Él puede ver en su pantalla los pacientes que están en espera y el tiempo estimado que les queda para ser atendidos. En caso
de que sea necesario, Rafa puede asignar manualmente un nivel de prioridad a un paciente en concreto bajo su propio criterio.
Gracias al triage inteligente, puede organizar los turnos de los pacientes de forma eficiente y rápida.


\subsection{Azucena Rodríguez Peralta}
\begin{itemize}
    \item \textbf{Contexto de uso: }
          \begin{itemize}
              \item \textbf{¿Cuándo utiliza el ordenador?: }  Usa el ordenador a diario
              \item \textbf{¿Dónde utiliza el ordenador?: } En casa y en el trabajo
              \item \textbf{¿Qué tipo de ordenador utiliza?: } El ordenador que le ofrecen en la consulta
          \end{itemize}
    \item \textbf{Misión: }
          \begin{itemize}
              \item \textbf{¿Para qué quiere utilizar nuestra aplicación?: } Para citar con antelación a los pacientes con mayor posibilidad de diagnóstico.
              \item \textbf{¿Qué espera encontrar en ella?: } Una web sencilla para saber cuando tiene cita con sus pacientes y los riesgos / síntomas que presentan.
          \end{itemize}
    \item \textbf{Motivación: }
          \begin{itemize}
              \item \textbf{¿Para cuándo quiere utilizarla?: } Para el trabajo
              \item \textbf{¿Por qué quiere alcanzar ese objetivo?: } Porque quiere diagnosticar posibles enfermedades lo antes posible
          \end{itemize}
    \item \textbf{Actitud hacia la tecnología: } Está muy acostumbrada a usar un ordenador
\end{itemize}

\subsection{Escenario Azucena Rodríguez Peralta}
Azucena es una mujer que ha estudiado la carrera de Enfermería y está especializada en matrona.
Acaba de mudarse a una nueva ciudad, donde comenzará a trabajar como matrona de consulta. Azucena se encarga
de las consultas a cerca del cáncer de cuello de útero donde se presupone un paciente sano, por lo que cita
a sus pacientes con bastante demora. Con frecuencia, Azucena tiene que atender a pacientes que muestran sintomas
alarmantes que dan indicios al diagnóstico del cáncer, por lo que propone usar un nuevo sistema de triage inteligente.
La web le permite a Azucena organizar las citas de sus pacientes y priorizar a los pacientes con mayor riesgo.
Cuando un paciente recibe una cita con Azucena, este tiene que rellenar un cuestionario con sus síntomas alarmantes.
Los pacientes con síntomas preocupantes son priorizados y citados con antelación. En caso de un diagnóstico precoz,
Azucena podrá manualmente cambiar el riesgo del paciente para no tenga tanta demora como los pacientes sanos.
Gracias al triage inteligente, Azucena puede llegar a diagnosticar enfermedades de forma precoz y rápida.


\section{Herramienta de CI/CD}
Una herramienta de integración continua y despliegue continua ayuda a un desarrollo iterativo, en el que se compruebe el trabajo constantemente; indispensable para el desarrollo ágil.
Por esta razón necesitamos una herramienta para automatizar las pruebas y la integración del código. Además, ayuda al tutor a corregir de una manera ágil e independiente, facilitando el trabajo en equipo de estudiantes y tutores.
La búsqueda de esta herramienta estará cribada por lo más estandarizado, software libre, las mejores prácticas y menos deuda técnica cause al proyecto.
Las posibles herramientas encontradas son Bitbucket, Github o Gitlab. Todas cumplen con los criterios de búsqueda, por lo que objetivamente podría ser cualquiera.
Finalmente, se ha escogido GitHub, debido a que es la más cómoda y fácil de usar para el tutor y el estudiante.
Como consecuencia de usar GitHub se "decide" escoger git, que es un sistema de control de versiones distribuido para manejar el proyecto con velocidad y eficiencia.

\section{Hitos}
\label{sec:hitos}
El desarrollo ágil debe ser incremental y para ello se debe dividir la carga de trabajo de diferentes hitos, esta carga
de trabajo corresponden a cada producto mínimamente viable (\textit{PMV}) y, a su vez, en diferentes capas de abstracción del producto.
Los hitos pueden definirse como metas a las que debemos de llegar durante el desarrollo del proyecto.
Los hitos son una buena forma de limitar los bloques de trabajo del desarrollo y su principal intención es que
al final de cada hito consigamos un producto mínimamente viable (\textit{PMV}) para que sea independiente e iterativo.
Durante el desarrollo de un hito surgen las \textit{issues}.

\subsection{Issues}
Las issues son problemas que van surgiendo durante el desarrollo del hito y
el conjunto de las soluciones de las issues de un hito deben conformar el PMV que satisfaga la necesidad del usuario.
Las issues deben hacer referencia a una historia de usuario, ya sea para solucionar un problema de la propia historia o para añadir una funcionalidad.
Todas las Issues que solucionaré durante el desarrollo para construir el sistema están descritas en GitHub. Cada issue debe tener mínimo un commit asociado a su
etiqueta y debe tener una pull request asociada para poder revisar el código.

\begin{figure}[!tb]
    \begin{center}
        \subfigure[Listado de issues.]{
            \includegraphics[width=0.9\linewidth]{logos/issues.png}
            \label{Imagen-Issues}}
        \subfigure[Listado de Pull Requests.]{
            \includegraphics[width=0.9\linewidth]{logos/pullrequest.png}
            \label{Imagen-PullRequests}}
        \label{Figura-Ciudades}
    \end{center}
\end{figure}


Los hitos que se deben de cumplir están reflejados en el GitHub y son los siguientes:

\subsection*{\href{https://github.com/RubenDelgadoPareja/TFG-Triage-Inteligente-Consulta-Medica/milestone/1}{Hito 0: Infraestructura inicial y documentación}}
\label{sb:hito0}

El primer PMV consiste en alcanzar un repositorio inicial donde comenzar a forjar toda la arquitectura del proyecto con una planificación inicial.
Para aceptar el PMV, el repositorio debe de contener:

\begin{itemize}
    \item{User Journeys.}
    \item{Hitos.}
    \item{Historias de Usuario.}
    \item{Comprobadores de gramática y ortografía.}
    \item{Comprobadores de compilación de LaTeX para la documentación.}
    \item{Documentar las secciones de Motivación, Objetivos y Planificación}
\end{itemize}

\subsection*{\href{https://github.com/RubenDelgadoPareja/TFG-Triage-Inteligente-Consulta-Medica/milestone/7}{Hito 1: Modelado del Problema con DDD}}
\label{sb:hito1}

Este PMV consiste en modular el dominio del problema para facilitar el diseño e implementación.
Para aceptar el PMV la memoria debe de recoger una sección específica sobre cómo se aplicará el DDD la cual incluya:

\begin{itemize}
    \item {Definición del dominio}
    \item {Definir el lenguaje ubicuo}
    \item {Identificar Entities, Value Objects, Aggregates}
    \item {Definir los Repositories y Services necesarios}
\end{itemize}


\subsection*{\href{https://github.com/RubenDelgadoPareja/TFG-Triage-Inteligente-Consulta-Medica/milestone/2}{Hito 2: Arquitectura básica del sistema de colas de citas}}

Este PMV consiste en diseñar las clases y funciones necesarias para crear citas, almacenarlas en colas y priorizarlas mediante una heurística.

Los criterios de aceptación de este PMV son los tests de cada clase y sus funciones, además de reflejar en la memoria las decisiones tomadas

\section{Historias de Usuarios}
Este concepto es fundamental en el desarrollo ágil, ya que, resume las necesidades del usuario.
Las historias de usuario se encuentran disgregando los aspectos atómicos del \textit{user journey} que aportan valor.
Deben de definirse en el dominio del problema, por lo que están relacionadas con la lógica de negocio.
Las historias de usuario describen la necesidad del usuario abstrayéndose de los detalles técnicos, ya que
el usuario no es un experto ni en informática ni en el dominio del problema.
Encontraremos las Historias de Usuario repartidas por los Hitos del desarrollo del software.

\noindent{Se han definido las siguientes historias de usuario:}

\subsection*{\href{https://github.com/RubenDelgadoPareja/TFG-Triage-Inteligente-Consulta-Medica/issues/19}{[HU-01] Triage Inteligente - Como sanitario de urgencias quiero que el triage inteligente asigne turnos de consulta automáticamente por riesgo.}}
Como sanitario de urgencias quiero un sistema de triage inteligente que se encargue de aplicar un cribado de riesgos a los pacientes.
El triage se encarga de evaluar el estado de salud del paciente a través de un formulario dónde se le pregunta al paciente el motivo de la consulta y
síntomas que padece, con esta información el sistema le asignará un riesgo de los cinco niveles establecidos, dependiendo del riesgo el paciente tendrá una demora
específica de tiempo para ser atendido. En caso de que supere el tiempo de demora y no sea atendido se podrá revaluar el riesgo del paciente.
El triage funcionará todo el día y puede tener supervisión de un sanitario en todo momento, este sanitario podría, según su criterio cambiar el riesgo del paciente.

\subsection*{\href{https://github.com/RubenDelgadoPareja/TFG-Triage-Inteligente-Consulta-Medica/issues/101}{[HU-02] Triage Inteligente - Como sanitario de citología quiero consultar los pacientes con mayores riesgos.}}
Como sanitario de citologías quiero un sistema de triage inteligente que se encargue de aplicar un cribado de riesgos a los pacientes con mayor riesgo.
El triage inteligente en citología estará enfocado a la propia citología, por lo que se tratará de un cuestionario acerca de posibles síntomas alarmantes
que den indicios de la enfermedad. En citología, la demora de consulta es muy alta, por lo que la diagnosis precoz es fundamental para el paciente.

\subsection*{\href{https://github.com/RubenDelgadoPareja/TFG-Triage-Inteligente-Consulta-Medica/issues/5}{[HU-03] Triage Inteligente - Como sanitario de citología quiero modificar el riesgo de un paciente.}}
Como sanitario que trabaja en citología que utiliza el triage inteligente, después de una consulta con un paciente que se le haya diagnosticado de manera precoz una enfermedad grave,
quiero aumentarle el riesgo para las siguientes consultas.

\section{Control de calidad}
\label{sc:control-calidad}
Para crear un software de calidad debemos de comprobar constantemente que lo desarrollado, para ello hemos usado la integración continua.
Necesitamos pasar todos los tests antes de poder mezclar el código con la rama principal, para ello empleamos GitHub Actions.
Los tests se ejecutan automáticamente cada vez que se suben nuevos cambios. Se han definido los siguientes tests:

\begin{itemize}
    \item{Compilación de la documentación de LaTeX.}
    \item{Corrector ortográfico y gramatical.}
    \item{Análisis estático de código de Typescript - ESLint.}
\end{itemize}

\section{Metodologías de diseño de la aplicación}
Durante todo el capítulo destaca la importancia de una planificación adecuada que permita adaptarse a los cambios de forma ágil y eficiente.
La metodología que se va a utilizar en el diseño busca una estructura que permita una implementación flexible y centrada en las necesidades del usuario.

El diseño del sistema es una parte fundamental del proceso de desarrollo de software, ya que proporciona la estructura sobre la cual se construirá el sistema.
En este proyecto, se ha elegido el Diseño Dirigido por el Dominio (DDD) \cite{domain-drive-design} como enfoque principal, en línea con el desarrollo ágil debido a que ambos se centran en el usuario.
Además, DDD es buena opción cuando los desarrolladores no conocen el dominio, porque mediante la comunicación constante con los expertos de dominio se puede llegar a un entendimiento común.
De esta comunicación surgirán los términos y conceptos que se utilizarán para describir el dominio del problema que se usarán para describir el dominio del problema, también conocido como lenguaje ubicuo.

\subsection{Buenas prácticas en la modelización del dominio}
La capa de dominio es el lugar donde se encuentra modelada el dominio y lógica del negocio en diferentes clases y funciones.
En Diseño Dirigido por el Dominio solemos encontrar modelos que representan entidades, objetos de valor, agregados, repositorios y servicios, sin embargo
el desarrollo ágil es quien nos guía en la implementación de estas clases y funciones. No desarrollaremos explícitamente estos conceptos si no es estrictamente necesario, se implementarán como un modelo más.
La metodología a seguir es la de desarrollar las clases necesarias para cumplir con las historias de usuario, de esta forma se evita la sobre abstracción y se consigue un desarrollo iterativo.
Esta forma de trabajo está respaldada por los principios del software \textit{DRY}: "Don't repeat yourself", \textit{YAGNI}: "You aren't gonna need it", o \textit{AHA}: "Avoid hasty abstractions".
En resumen, estos principios nos dicen que no debemos de abstraer antes de tiempo, que no debemos de repetir código y que no debemos de hacer abstracciones innecesarias.

A pesar de lo mencionado anteriormente, debido a como se ha ido resolviendo el desarrollo  del código, se ha hecho una separación por responsabilidades en los modelos.
Separando la lógica interna del modelo con los parámetros de creación del propio modelo, para facilitar el cambio y la adaptación del modelo a las necesidades del usuario.
De esta manera cuando se agregue nuevo código, solamente se modificará esta interfaz y no la lógica ya desarrollada.
Esto es fundamental para el desarrollo ágil, ya que permite adaptarse a los cambios de manera rápida y eficiente.

Para modelar el dominio y la lógica se ha requerido conocer conceptos y datos del dominio, por lo que se ha requerido de la colaboración de expertos en el dominio.
En este caso, contaba con un experto cercano al dominio, mi hermana que es enfermera. Además, he conseguido información de expertos en este \href{https://scielo.isciii.es/scielo.php?script=sci_arttext&pid=S1137-66272010000200008}{artículo de un sistema de triage navarro}.



% Implementación

\chapter{Implementación}

En este capítulo se pretende reflejar las herramientas usadas durante toda la implementación y
se argumentará por qué se han tomado estas decisiones, al igual que en el capítulo de metodología \ref{ch:metodología} explicamos
la planificación y forma de trabajo.

El desarrollo del software se ha dividido en hitos \ref*{sec:hitos}. Cada hito representa un PMV que se publicará.
PMV son las siglas de Producto Mínimo Viable, es la versión más simple y básico del producto que proporciona valor al cliente.

Los \href{https://github.com/RubenDelgadoPareja/TFG-Triage-Inteligente-Consulta-Medica/milestones}{hitos} han sido definidos en GitHub.
Cada uno de ellos tiene asignados \href{https://github.com/RubenDelgadoPareja/TFG-Triage-Inteligente-Consulta-Medica/issues}{\textit{issues}} que se corresponden
con los distintos problemas que han ido surgiendo durante el desarrollado, cómo los hemos solucionado, qué herramientas se han utilizado y por qué.
La solución de las issues asignadas al hito conforman el llamado producto mínimo viable (\textit{PMV}) del hito.

\section{Hito 0: Infraestructura y documentación inicial}

Lo he llamado hito número 0 porque es el punto de partida del proyecto y no aporta un valor real al cliente, sino a nosotros como desarrolladores.
El PMV del hito 0 \ref{sb:hito0} es un repositorio inicial formado por una documentación y un conjunto de herramientas configuradas para el desarrollo del proyecto.
Las herramientas configuradas durante este hito son las siguientes:

\subsection{Entorno de desarrollo}
Antes de poder empezar con el hito 0 debemos de plantearnos qué entorno usar como base de todo el trabajo.
Necesitamos una herramienta donde comenzar el desarrollo del proyecto y configuraciones de herramientas.
Los criterios de búsqueda de este entorno será el estándar, el software libre, multilenguaje, de calidad, de fácil empleo y configuración, por orden de prioridad.
Después de una pequeña búsqueda sobre los entornos de desarrollo, no existe un estándar, pero sí que hay un grupo de más usados, los cuales son: VS Code, IntelliJ IDEA, PyCharm, Eclipse, Netbeans.
De esta lista he seleccionado finalmente VS Code, es el único que cumple con la mayoría de criterios. La mayoría de entornos depende del lenguaje de programación, pero al no haberlo
decidido aún elijo VS Code, además es software libre.

\subsection{Integración continua}
La integración continua es una práctica de desarrollo de software que tiene como objetivo principal mejorar la calidad del código, aumentar la eficiencia y agilizar el proceso de entrega de software.
Debemos encontrar una herramienta que nos permita automatizar los procesos de integración en la nube, teniendo en cuenta que sea gratuita para código abierto, fácil de configurar y sea compatible con el repositorio del proyecto en GitHub.
Las herramientas que cumplen con estos requisitos son Travis CI, CircleCI, GitHub Actions. La opción elegida es GitHub Actions, es la herramienta perfecta en este caso, ya que es gratuita, fácil de configurar y está integrada en el propio GitHub.
Esto nos ayuda a poder relanzar los flujos de trabajo desde la propia interfaz de GitHub.

Gracias a esta metodología podemos automatizar la ejecución de herramientas de calidad, de esta manera comprobaremos antes de ciertas acciones con los hooks de GitHub que todo está correctamente.
La potencia de esta herramienta es ejecutar diferentes flujos de trabajo en paralelo o \textit{workflow} en el que hacen diferentes comprobaciones garantizando la calidad en todo momento.
Cada workflow se ejecuta en cada push a la rama principal, estos workflow podrán ejecutar diferentes tareas dependiendo de la configuración.
A continuación sigo con las herramientas que aportan calidad tanto al código como a la documentación.

\subsection{Herramientas de documentación}
La memoria es muy importante en el desarrollo de un proyecto, ya que es el reflejo de todo el trabajo realizado y es necesario para la evaluación del proyecto.
Debemos de encontrar una herramienta que nos permita escribir la memoria de forma cómoda y eficiente. Además de revisar constantemente que lo escrito esté correcto.

\subsubsection{Generación de la memoria}
La memoria se ha escrito empleando el lenguaje LaTeX, que es un sistema de composición de documentos bastante empleado
en el ámbito académico y científico, es necesario para poder generar la documentación final, agrupando cada capítulo, sección y
gráfica. Para facilitar la construcción de la memoria se ha utilizado un gestor de tareas con la finalidad de automatizar la generación de la memoria.
Las opciones que se han tenido en cuenta son Make, CMake, Gradle, Rake o Gulp. Se ha escogido Make porque es la más sencilla de configurar y la más extendida, el resto
están ligados a un lenguaje de programación en concreto y no se ajustan a las necesidades del proyecto, ya que añadiría una complejidad innecesaria.

\subsubsection{Corrector de la memoria}
Para garantizar la calidad de la documentación dentro del desarrollo ágil, es necesario un corrector ortográfico y gramatical.
La herramienta \href{https://github.com/sylvainhalle/textidote}{TeXtidote} nos permite corregir la ortografía, gramática y semántica de la documentación escrita.
Se puede ejecutar la corrección a través del comando \textit{check} del Makefile, generando un borrador con la documentación marcando los errores y posibles cambios.
Además, se ha añadido este comando a los hooks de GitHub para ejecutarlo antes de cada commit.

Con la intención de continuar con el paradigma de la calidad y las buenas prácticas claves para el desarrollo ágil, se ha configurado el entorno con
una tarea en el Makefile para ejecutar para cada herramienta, \textit{all} para generar la memoria y \textit{check} para revisarla.

\subsection{Herramientas de desarrollo}
El código fuente debe de garantizar la calidad, la eficiencia y mantenimiento del código a largo plazo. Para ello necesitamos de la ayuda de diferentes herramientas
que nos ayuden a detectar errores durante y después el desarrollo. Pero estas herramientas dependen del lenguaje de programación escogido, por lo que debemos de definirlo antes.

\subsubsection{Lenguaje de programación}
Primero debemos de escoger un lenguaje de programación para el desarrollo del proyecto, el cual cumpla una serie de criterios para que sea adecuado.
Para elegir correctamente deberíamos de hacer un estudio del alcance y objetivos del proyecto, los recursos y habilidades disponibles, la experiencia del desarrollador etc.
Los criterios que para elegir el lenguaje son la experiencia del programador, la portabilidad, legibilidad, seguridad del código y la comunidad.
Los lenguajes que cumplen estos criterios son los siguientes: JavaScript, TypeScript, Java, Ruby, C++, Python. Inicialmente, porque son los lenguajes con los que he trabajado.
Como el proyecto será una aplicación web buscamos lenguajes portables con los diferentes navegadores actuales, destaca JavaScript porque los navegadores desde 2012 soportan ECMAScript 5.1 que está basado en JavaScript.
Además, como TypeScript es un superconjunto de JavaScript, también sería buena opción en ese aspecto y gracias al tipado fuerte que tiene aporta seguridad y legibilidad al código, mejorando la calidad del software.
Los lenguajes como Java y Python también serían buenas opciones, pero finalmente se han descartado por la poca experiencia y la falta de recursos para comenzar a desarrollar en ellos.
Por tanto, el lenguaje de programación escogido es TypeScript;

\subsubsection{Análisis estático}
Para el desarrollo ágil es imprescindible garantizar la calidad, la eficiencia y sostenibilidad del código a largo plazo.
En TypeScript, a pesar de tener un tipado fuerte, es posible que se cometan errores que no se detecten hasta que se ejecute el código; por ello es necesario un análisis estático del código.
Buscamos una herramienta rápida y sencilla de configurar que detecte errores de programación en TypeScript.
Las posibles herramientas a escoger son ESLint o TSLint, porque son las únicas herramientas especifica para el lenguaje escogido, TypeScript. Aunque, TSLint está obsoleta y no tiene mantenimiento.
Nos quedamos con ESLint que te permite aplicar reglas personalizadas, reglas predefinidas y configuraciones de estilo para garantizar que tu código sea consistente y libre de problemas.

\subsubsection{Test unitarios.}
A medida que se va desarrollando el software debemos de garantizar que el código es de calidad y funcional, para ello es necesario probar todo el código, ya que código sin
testear es código que no funciona. Además, cuando el código supera los tests nos aseguramos de que realmente aporta un valor al cliente.
Para probar el código debemos de escoger una herramienta que nos permita realizar test unitarios en TypeScript.
Los criterios para escoger la herramienta son la compatibilidad con Typescript, que sea de software libre, fácil de configurar, que se pueda configurar con la integración continua,
que ayude a mockear los datos necesarios para los tests.
Las opciones que se han encontrado a valorar son las siguientes: Jest, Mocha, Cypress, Jasmine y AVA. De todas estas opciones, se ha escogido Jest sobre el resto debido a que es la
mejor herramienta en los criterios respecto al resto. Mocha tiene una documentación mala y conlleva más tiempo de preparar para usarlo, ya que la configuracion con Typescript es manual.
AVA al igual que Mocha necesita una configuración especial para emplearlo junto a TypeScript. Jasmine es en general buena opción, pero es la interfaz de usuario es más compleja.
Cypress realmente es una herramienta para hacer test E2E y depende del navegador. Quizás en un futuro se podría utilizar para hacer test E2E, no obstante no para test unitarios.
Jest tiene una gran comunidad, buena documentación, fácil integración con Typescript, ayuda a mockear los datos necesarios para los tests y trae espiadores integrados.
Además, tiene una cobertura integrada en el código para saber cuánto porcentaje del código está testado.

\subsubsection*{Framework de desarrollo}
Durante el desarrollo del proyecto se ha tomado la decisión de emplear un framework de desarrollo para facilitar la creación de la aplicación web.
Utilizar un framework acelera el desarrollo al proporcionar una estructura predefinida, buenas prácticas y herramientas integradas, lo que mejora la calidad del código, la seguridad y la mantenibilidad.
Además, reduce el tiempo de desarrollo y facilita la escalabilidad, haciendo el proceso más eficiente y organizado.
Los criterios de búsqueda para el framework serán una configuración sencilla, fácil de utilizar, que soporte Typescript y Node.js, que se adapte a la situación del proyecto y que sea pragmático.
He de explicar que el desarrollo del proyecto se centrará en la lógica de negocio y en la parte del backend, por lo que los framework serán orientados a este tipo de desarrollo.
Las opciones que se han barajado son las siguientes: Express, NestJS, Koa, Hapi. Express se ha descartado por ser un framework muy sencillo y no aporta tanto como los demás.
Hapi se descarta, ya que, está enfocado a los plugins, lo que no se ajusta a las necesidades del proyecto. Koa se descarta, puesto que, utiliza una arquitectura de middleware y lo mejor sería una arquitectura modular.
Finalmente, se ha escogido NestJS por ser el framework que mejor se amolda a nuestro proyecto.


\subsubsection*{Base de datos}
Con la intención de continuar con el paradigma de la calidad y las buenas prácticas claves para el desarrollo ágil, se ha optado
por implementar una base de datos en el proyecto. La persistencia de los datos, garantiza que los datos no se pierdan y se puedan recuperar en cualquier momento.
Además, ayuda a que el proyecto sea más escalable y eficiente. Los criterios que se han escogido para seleccionar la base de datos han sido facilidad de configuración.
La elección de una base de datos es algo que no debería hacerse a la ligera, ya que implica un nuevo conjunto de tecnologías y herramientas que pueden ser complicadas de manejar.
Las opciones que se han barajado son las siguientes: MySQL, PostgreSQL, MongoDB, SQLite. Finalmente, se ha escogido PostgreSQL.
PostgreSQL es generalmente la opción más versátil y poderosa si buscas un equilibrio entre rendimiento, escalabilidad, y funcionalidad avanzada.


% Presupuesto

% Conclusiones

\chapter{Conclusiones y trabajos futuros}



% Trabajos futuros



\newpage
\bibliography{bibliografia}
\bibliographystyle{plain}

\end{document}

