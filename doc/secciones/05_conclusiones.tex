\chapter{Conclusiones y trabajos futuros}

\section{Conclusiones}

El desarrollo de esta API de triage inteligente para urgencias ha permitido avanzar en la automatización de la gestión de pacientes en situaciones críticas, cumpliendo con los objetivos propuestos.
La solución implementada no solo contribuye a optimizar los tiempos de espera mediante la asignación dinámica de prioridades basada en la gravedad del paciente, sino que también ayuda a paliar el problema de saturación en las áreas de urgencias hospitalarias.

Entre los principales logros del proyecto destacan los siguientes puntos:

\begin{itemize}
    \item Se ha desarrollado una API REST que integra el registro de pacientes, la gestión de formularios médicos y la asignación de turnos con prioridad basada en el triage, facilitando la comunicación entre los sistemas de urgencias.
    \item La API es interoperable y flexible, y utilizando estándares que garantizan su escalabilidad y facilidad de uso.
    \item Se ha optimizado el proceso de triage gracias a la implementación de algoritmos que priorizan en tiempo real a los pacientes según su riesgo, mejorando los flujos de trabajo en los hospitales y minimizando el impacto de la saturación.
\end{itemize}

Personalmente, la realización de este trabajo ha permitido poner en práctica los conocimientos adquiridos durante el grado y me ha brindado la oportunidad de profundizar en áreas clave como la gestión de emergencias, el diseño de APIs REST y la optimización de sistemas hospitalarios.
Además, resalto la importancia de desarrollar soluciones tecnológicas que puedan integrarse fácilmente con otras plataformas.
Finalmente, subrayo la relevancia de utilizar herramientas abiertas y accesibles en el desarrollo de este tipo de soluciones, ya que facilita futuras mejoras y contribuciones por parte de la comunidad.

\section{Trabajos futuros}

Si bien se han logrado avances significativos con este proyecto, aún queda mucho por hacer para alcanzar una automatización completa y eficiente del proceso de triage en urgencias. A continuación, se proponen algunas líneas de trabajo para futuras mejoras:

\begin{itemize}
    \item \textbf{Ampliación de la integración de datos clínicos}: Se podría integrar la API con sistemas avanzados de historia clínica electrónica (EHR) para obtener más información sobre el historial del paciente en tiempo real, lo que permitiría una evaluación más precisa en el triage.
    \item \textbf{Optimización de algoritmos de priorización}: Investigar nuevas técnicas de aprendizaje automático para mejorar la precisión y rapidez de la asignación de turnos, considerando factores como las predicciones de recursos necesarios o la probabilidad de complicaciones graves.
    \item \textbf{Implementación de módulos predictivos}: Desarrollar módulos adicionales que permitan predecir la carga de trabajo en urgencias según patrones históricos o análisis en tiempo real, anticipándose a posibles picos de saturación y ajustando los recursos en consecuencia.
    \item \textbf{Mejoras en la experiencia de usuario}: Facilitar la usabilidad de la API tanto para el personal médico como para los administradores de sistemas, mediante interfaces gráficas que visualicen en tiempo real el flujo de pacientes y los tiempos de espera estimados.
    \item \textbf{Seguridad y protección de datos}: Ampliar las medidas de seguridad en la transmisión y almacenamiento de información médica, garantizando el cumplimiento de normativas como el RGPD (Reglamento General de Protección de Datos) y mejorando la encriptación de datos sensibles.
\end{itemize}

Estos trabajos futuros permitirían que el sistema no solo optimice la gestión de urgencias, sino que también brinde soporte a los profesionales de la salud en la toma de decisiones críticas, mejorando la calidad de la atención y la eficiencia hospitalaria.
