\chapter{Planificación}

Para poder hablar de la implementación primero debemos de establecer cuál es la metodología y el control de calidad 
de la solución propuesta

\section{Metodología}

Buscamos un tipo de metodología que se centre en el usuario y nos aporte retroalimentación.
La metodología ágil nos permite adaptarnos al cambio de requisitios haciendo uso de las iteraciones y 
evitando caer en una documentación excesiva llena de diagramas y requisitos innecesarios.
La metodología ágil construye el sistema por medio de aportaciones frecuentes de código que aportan valor 
al usuario, es necesario organizar las funcionalidades en bloques y que los cambios siempre surgen de una 
necesidad del usuario
He de mencionar que todos los commits deben de hacer referencia a las tareas preestablecidas, por lo tanto 
podremos saber cuando leyamos el código hacia que historia de usuario o issue se refiere

\subsection{Control de versiones}
Para el control de versiones utilizo git y GitHub como repositorio para hacer un seguimiento de todas las tareas
y poder observar el desarrollo en conjunto del proyecto 

\subsection{Hitos}
Los hitos pueden definirse como metas a las que debemos de llegar durante el desarrollo del proyecto.
Los hitos son una buena forma de limitar los bloques de trabajo del desarrollo y su principal intención es que 
al final de cada hito consiguamos un producto software funcional para que sea independiente e iterativo 
Los hitos que se deben de cumplir están reflejados en el GitHub y son los siguientes: 
\begin{itemize}
    \item Documentación
    \item Arquitectura inicial del Triage Inteligente 
    \item Testing del Triage Inteligente inicial 


\subsection{Historias de Usuarios}
Las historias de usuario son la unidad mínima de funcionalidad que el usuario necesita del sistema, es decir, 
expresa lo que quiere el usuario del sistema por esto lo hace tan importante, se centra completamente en el 
usuario. Encontraremos las Historias de Usuario repartidas por los Hitos del desarrollo del software 

\subsection{Issues}
Las Issues son simples tareas que hay que realizar para completar los Hitos. 
Una historia de Usuario podría ser un conjunto de Sssues , aunque en GitHub las he englobado en una sola Issue
Todas las Issues que solucionaré durante el desarrollo para construir el sistema están descritas en GitHub 


\subsection{Control de calidad}