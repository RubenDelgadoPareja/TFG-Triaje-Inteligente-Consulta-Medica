\chapter{Introducción}

Este proyecto es software libre, y está liberado con la licencia \cite{gplv3}.

\section{Motivación}
Debido a la crisis económica del 2008 la sanidad española no ha parado de recibir recortes desde entonces, puede observarse en este
\href{https://www.consalud.es/politica/decada-recortes-2009-2018-efectos-infrafinanciacion-sanidad_87083_102.html}{artículo} donde
se estudia la financiación de la sanidad desde entonces. Esto provoca que se disminuya el número de sanitarios en España, 
esta situación que se lleva dando desde hace más de una década produce una sobrecarga de trabajo para los sanitarios y un retardo 
en todas las listas de esperas, tanto en cirugias como en la atención primaria. 
Con la llegada de la pandemia y del covid-19 todo empeoró llegando a colapsar totalmente la atención primaria debido a la demanda
de citas previas y a la escasez de profesionales enfocados a este servicio ya que también desde hace años los sanitarios se aglomeran
en los Hospitales por mejores situaciones laborales y salarios. En las siguientes links podemos observar la situación que he descrito.
\begin{itemize}
    \item \url{https://elpais.com/sociedad/2021-02-08/la-pandemia-sume-a-la-atencion-primaria-en-una-saturacion-permanente.html} 
    \item \url{https://www.lavanguardia.com/vida/20211225/7952109/atencion-primaria-colapso-pacientes-medico-dias-espera.html}
    \item \url{https://www.lavozdegalicia.es/noticia/galicia/2022/01/16/dejando-atender-casos/0003_202201G16P2993.htm}

El problema que trato de abordar se escapa a mis posibilidades, sin embargo mediante la herramienta que voy a desarrollar 
durante el proyecto podré aportar una gran ayuda para los sanitarios centrados en la atención primaria.
Mi principal motivación es aportar mediante el uso de sistemas informáticos mi grano de arena en este gran problema con el que 
tenemos que lidiar a día de hoy, construyendo una aplicación de citas médicas el cuál tiene integrado un sistema de triage. 
Además estará centrado para el uso de todas las personas y sanitarios, haciendo incapié en la accesibilidad de la app.

\section{Objetivos}

\textbf{Crear el triage inteligente de las consultas médicas}. El objetivo del sistema es minimizar las repercusiones negativas 
de las listas de espera saturadas ya que realmente es un problema económico de la sanidad pública. De forma que el triage de consultas
priorizá las personas con mayores riesgos en las que las consecuencias de la saturación de colas puedan ser críticas.

El sistema informática consistirá en un triage de consultas médicas ; el triage en el ámbito de la sanidad es un proceso que permite gestionar
el flujo de los pacientes teniendo en cuenta el riesgo cuando la demanda y necesidades clínicas superan los recursos, es muy importante en urgencias clínicas
\href{https://scielo.isciii.es/scielo.php?script=sci_arttext&pid=S1137-66272010000200008}{Aquí} se puede encontrar más información
El triage tiene un gran potencial que no se aprovecha en los sistemas de citas de la sanidad actual.

\subsection{Proyecto a futuro}
El principal proyecto a futuro sería integrar el triage como tal a los sistemas instaurados en la sanidad como si se tratase de 
una librería de código. Además puedrían desarrollarse los siguientes puntos:

\begin{itemize}
    \item Aplicar la aplicación para consultas especializadas (Neurología, Urología, Ginecología \dots) 
    \item Construir e integrar un plugin para acceder a las recetas de los medicamentos de una consulta
    \item Crear consultas automatizadas para enfermedades crónicas
