\chapter{Introducción}

Este proyecto es software libre, y está liberado con la licencia \cite{gplv3}.

\section{Motivación}
Desde el inicio de la pandemia los sistemas de consultas médicas se han ido quedando saturados debido
a la sobrecarga de demanda de este servicio provocando unas listas de esperas desmesuradas para las personas

Este es el principal problema que voy a tratar de ponerle fin con mi proyecto, el cual consiste en diseñar y 
desarollar un sistema de triage para las consultas que permita priorizar las personas con mayor riesgo de salud.

Este sistema estará enfocada a todo el mundo que haga uso de la sanidad pública, por lo que debe estar preparada
para personas de todo tipo, haciendo incapié en personas mayores 

\section{Objetivos}

\subsection{Principal}

\textbf{Crear el triage inteligente de las consultas médicas}. Esta es la parte más compleja del sistema ya que
conlleva crear un sistemas de colas de citas priorizada, las cualés se ordenarán mediante una heuristica que tendrá en cuenta
el riesgo de cada usuario. Además debe de estar integrada en una aplicación accesible para todos los usuarios 
asegurando la calidad del código 

\subsection{Opcionales}
Los objetivos secundarios podrían ser:

\begin{itemize}
    \item Almacenar el historial médico de una consulta en la aplicación
    \item Modificar el riesgo de los usuario 

\subsection{Proyecto a futuro}
Los objetivos que se quedarán para el futuro del proyecto serán: 

\begin{itemize}
    \item Aplicar la aplicación para consultas especializadas 
    \item Construir un plugin para acceder a las recetas de los medicamentos de una consulta
    \item Crear consultas automatizadas para enfermedades crónicas
