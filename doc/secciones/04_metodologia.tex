\chapter{Metodología}

Para comenzar el desarrollo de un proyecto primero necesitamos una planificación inicial de la que partir.

\section{Planificación}

La planificación que se va a utilizar está basada en los {\href{https://agilemanifesto.org/iso/es/principles.html}{principios del manifiesto ágil}}
Buscamos un tipo de planificación que se centre en el usuario, para que el propio usuario nos aporte retroalimentación.
Una planificación ágil nos permite adaptarnos al cambio de requisitos haciendo uso de las iteraciones y
evitando caer en una documentación excesiva llena de diagramas y requisitos innecesarios.
El desarrollo ágil construye el sistema por medio de aportaciones frecuentes de código que aportan valor
al usuario, es necesario organizar las funcionalidades en bloques y que los cambios siempre surgen de una
necesidad del usuario.
Para cualquier tipo de planificación es indispensable una herramienta de control de versiones, para ello voy a emplear  git
y GitHub como repositorio para hacer un seguimiento de todas las tareas.
He de mencionar que todos los commits deben de hacer referencia a las tareas preestablecidas, por lo tanto,
podremos saber cuando leamos el código hacía que historia de usuario o issue se refiere
y poder observar el desarrollo en conjunto del proyecto

\section{Hitos}
Los hitos pueden definirse como metas a las que debemos de llegar durante el desarrollo del proyecto.
Los hitos son una buena forma de limitar los bloques de trabajo del desarrollo y su principal intención es que
al final de cada hito consigamos un producto mínimamente viable (\textit{PMV}) para que sea independiente e iterativo
Los hitos que se deben de cumplir están reflejados en el GitHub y son los siguientes:

\subsection*{\href{https://github.com/RubenDelgadoPareja/TFG-Triage-Inteligente-Consulta-Medica/milestone/1}{Hito 0: Infraestructura inicial y documentación}}

El primer PMV consiste en alcanzar un repositorio inicial donde comenzar a forjar toda la arquitectura del proyecto con una planificación inicial.
Para aceptar el PMV, el repositorio debe de contener:

\begin{itemize}
    \item{User Journeys.}
    \item{Hitos.}
    \item{Historias de Usuario.}
    \item{Comprobadores de gramática y ortografía.}
    \item{Comprobadores de compilación de LaTeX para la documentación.}
    \item{Documentar las secciones de Motivación, Objetivos y Planificación}
\end{itemize}

\subsection*{\href{https://github.com/RubenDelgadoPareja/TFG-Triage-Inteligente-Consulta-Medica/milestone/7}{Hito 1: Modelado del Problema con DDD}}

Este PMV consiste en modular el dominio del problema para facilitar el diseño e implementación.
Para aceptar el PMV la memoria debe de recoger una sección específica sobre cómo se aplicará el DDD la cual incluya:

\begin{itemize}
    \item {Definición del dominio}
    \item {Definir el lenguaje ubicuo}
    \item {Identificar Entities, Value Objects, Aggregates}
    \item {Definir los Repositories y Services necesarios}
\end{itemize}


\subsection*{\href{https://github.com/RubenDelgadoPareja/TFG-Triage-Inteligente-Consulta-Medica/milestone/2}{Hito 2: Arquitectura básica del sistema de colas de citas}}

Este PMV consiste en diseñar las clases y funciones necesarias para crear citas, almacenarlas en colas y priorizarlas mediante una heurística.

Los criterios de aceptación de este PMV son los tests de cada clase y sus funciones, además de reflejar en la memoria las decisiones tomadas

\section{Historias de Usuarios}
Las historias de usuario son la unidad mínima de funcionalidad que el usuario necesita del sistema, es decir,
expresa lo que quiere el usuario del sistema por esto lo hace tan importante, se centra completamente en el
usuario. Encontraremos las Historias de Usuario repartidas por los Hitos del desarrollo del software.

\noindent{Se han definido las siguientes historias de usuario:}

\subsection*{\href{https://github.com/RubenDelgadoPareja/TFG-Triage-Inteligente-Consulta-Medica/issues/19}{[HU-01] Triage Inteligente - Priorizar una cita.}}

Como paciente quiero tener prioridad en el orden de citas médicas.

\subsection*{\href{https://github.com/RubenDelgadoPareja/TFG-Triage-Inteligente-Consulta-Medica/issues/5}{[HU-02] Triage Inteligente - Establecer riesgo.}}

Como sanitario quiero asignarle una prioridad o riesgo a un paciente

\section{Issues}
Las Issues son simples tareas que hay que realizar para completar los Hitos.
Una historia de Usuario podría ser un conjunto de Issues, aunque en GitHub las he englobado en una sola Issue
Todas las Issues que solucionaré durante el desarrollo para construir el sistema están descritas en GitHub


\section{Control de calidad}

