\thispagestyle{empty}

\begin{center}
	{\large\bfseries Triage Inteligente de Consulta Médica \\ Sistema de citas priorizado por el riesgo del paciente }\\
\end{center}
\begin{center}
	Rubén Delgado Pareja\\
\end{center}

%\vspace{0.7cm}

\vspace{0.5cm}
\noindent\textbf{Palabras clave}: \textit{Software libre, Triage, Metodología Ágil, Diseño Dirigido por el Dominio,}
\vspace{0.7cm}

\noindent\textbf{Resumen}\\
Debido al precario estado de la sanidad española las colas de citas médicas y urgencias están saturadas. Aprovechando el potencial de los sistemas informáticos se pretende minimizar las
consecuencias negativas de estas largas esperas usando el método del triage para priorizar a los pacientes más graves. Para ello se propone un sistema de citas médicas que asigna turnos en función de la gravedad del paciente
Siguiendo la metodología ágil se ha implementado una API Rest que permita registrar pacientes, almacenar su información médica a través de un formulario basado en el protocolo de triage y asignarles un turno en función de su nivel de prioridad.

\cleardoublepage

\begin{center}
	{\large\bfseries Intelligent Triage for Medical Consultation \\ Appointment system prioritized by patient risk}\\
\end{center}
\begin{center}
	Rubén Delgado Pareja\\
\end{center}
\vspace{0.5cm}
\noindent\textbf{Keywords}: \textit{Open source, Triaje, Agile methodology, Domain Driven Design}, \textit{floss}
\vspace{0.7cm}

\noindent\textbf{Abstract}\\
Due to the precarious state of the Spanish healthcare system, queues for medical appointments and emergencies are saturated. By exploiting the potential of IT systems, the aim is to minimise the negative consequences of long waiting times by using the
negative consequences of these long waits by using the triage method to prioritise the most serious patients. To this end, a system of medical appointments is proposed that assigns shifts according to the severity of the patient.
Following the agile methodology, an API Rest has been implemented to register patients, store their medical information through a form based on the triage protocol and assign them a shift according to their priority level.

\cleardoublepage

\thispagestyle{empty}

\noindent\rule[-1ex]{\textwidth}{2pt}\\[4.5ex]

D. \textbf{Juan Julián Merelo Guervós}, Profesor del departamento de Arquitectura y Tecnología de Computadores.

\vspace{0.5cm}

\textbf{Informo:}

\vspace{0.5cm}

Que el presente trabajo, titulado \textit{\textbf{Triage Inteligente de Consulta Médica. Sistema de citas priorizado por el riesgo del paciente}},
ha sido realizado bajo mi supervisión por \textbf{Rubén Delgado Pareja}, y autorizo la defensa de dicho trabajo ante el tribunal
que corresponda.

\vspace{0.5cm}

Y para que conste, expiden y firman el presente informe en Granada a Septiembre de 2024.

\vspace{1cm}

\textbf{El/la director(a)/es: }

\vspace{5cm}

\noindent \textbf{Juan Julián Merelo Guervós}

\chapter*{Agradecimientos}

Primero de todo, quiero agradecérmelo a mi mismo, por ser capaz de llevar a cabo este proyecto. A pesar del poco tiempo que he podido invertirle por temas laborales.
Después quería agradecer a toda la comunidad de informáticos, ya que son la mejor comunidad que he conocido jamás, de la mano también debo de agradecer al software libre.
Sin el software libre estaríamos limitando el conocimiento y la tecnología solamente a personas con recursos económicos.
Finalmente, agradecer a mi familia, amigos y compañeros de trabajo por su apoyo y comprensión durante la realización de este proyecto.
